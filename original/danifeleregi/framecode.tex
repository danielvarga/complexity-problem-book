\documentclass[12pt]{book}
\usepackage[utf8]{inputenc}
\usepackage{t1enc} % hogy a belső kódkészlet kibővüljön. Így az ő betűt egy betűnek tekinti, és nem egy o betűnek, amire rápakoltunk ékezeteket.
\usepackage{graphicx} % hogy lehessen .eps vagy .pdf képeket berakni az \includegraphics{filenév.eps} paranccsal.
%\usepackage[magyar]{babel} % hogy magyar elválasztás legyen.
\usepackage{ae} % hogy a pdflatex után keletkező .pdf file-ban csinos karakterek legyenek. % Hé, nem vágja ez haza a kereshetőséget?
%% \usepackage{hyperref} % hogy a .pdf file-ban lehessenek klikkelhető hivatkozások.
\usepackage{amsmath} % Mátrixokhoz például.
\usepackage{amssymb} % $\mathbb{Z}$ miatt.
\usepackage{xspace} % A makrók utáni space korrekt kezelése miatt.
\usepackage{ifthen} % A feltételes vezérléshez.
\usepackage{color}

%------------------------
% A struktúrát meghatározó #define-ok:

% Ha true, akkor egyetlen része lesz a könyvnek, és minden feladat után rögtön ott a hint és a megoldás.
% Ha false, akkor a könyv első része a feladatok, aztán a hintek magányosan, aztán a feladatok a megoldásokkal.
% Ehhez háromszor inkludálom be a dokumentumot.
\newboolean{compactForm}
\setboolean{compactForm}{true}

% Megjelenítse-e az ideiglenes kommenteket (\wrk):
\newboolean{showComments}
\setboolean{showComments}{true}

%------------------------

% Fejezet elején és végén. Ha a chapter paranccsal összevonva új parancsot hoznánk létre, akkor egyáltalán nem kellene. De egyelőre ne bántsuk.
\newcommand{\sectionstart}{ \setcounter{problem}{0} }
\newcommand{\sectionend}{}


% Nehézségi szint:
\newcommand{\hard}{(*) }
\newcommand{\veryhard}{(**) }
\newcommand{\easy}{($\heartsuit$) } % {($\flat$) }


% Bonyolultsági osztály:
\newcommand{\cl}[1]{\mbox{\ensuremath{\mathbf{#1}}}\xspace}

% Fontos konkrét bonyolultsági osztályok:
\renewcommand{\P}{\cl{P}}
\newcommand{\NP}{\cl{NP}}
\newcommand{\Ppoly}{\cl{P/poly}}
\newcommand{\PSPACE}{\cl{PSPACE}}
\newcommand{\BPP}{\cl{BPP}}
\newcommand{\RP}{\cl{RP}}
\newcommand{\ZPP}{\cl{ZPP}}
\newcommand{\EXP}{\cl{EXP}}
\newcommand{\NEXP}{\cl{NEXP}}
\newcommand{\DTIME}{\cl{DTIME}}
\newcommand{\NTIME}{\cl{NTIME}}
\newcommand{\SPACE}{\cl{SPACE}}
\newcommand{\SIZE}{\cl{SIZE}}
\newcommand{\E}{\cl{E}}
\newcommand{\NE}{\cl{NE}}
\newcommand{\LOGSPACE}{\cl{LOGSPACE}}
\newcommand{\NL}{\cl{NL}}
\newcommand{\coNP}{\cl{co-NP}}
\newcommand{\PH}{\cl{PH}}
\newcommand{\ACnull}{\cl{AC^0}}
\newcommand{\Sigmatwo}{\cl{\Sigma_2}}
\newcommand{\TALLY}{\cl{TALLY}}
\newcommand{\SPARSE}{\cl{SPARSE}}

% Nyelv:
\newcommand{\la}[1]{\mbox{\sc{#1}}\xspace}

% Konkrét nyelvek:
\newcommand{\SAT}{\la{SAT}}
\newcommand{\Language}{\la{L}} % Na ő éppenhogy nem konkrét. És a nevének kellemetlen tulajdonsága, hogy ösztönösen ,,a \\Language''-et írok.


% A részfeladatokat (a), (b), (c) számozzuk:
\renewcommand{\theenumi}{\alph{enumi}}
\renewcommand{\labelenumi}{(\theenumi)}

\newcommand{\remark}[1]{\textbf{Megjegyzés:} #1}

\newcommand{\nothing}{\vspace{0mm}}

\newtheorem{problem}{Feladat}

% Olyan referencia, ami látszik is az olvasó számára. Egyelőre nem csináltam meg, de tervezem azt a fajta referenciát, ami csak a számomra látszik, hogy konkrétan honnan vettem egy-egy folklór eredményt.
\newcommand{\reff}[1]{(#1)}

% Ne pattogjon páros-páratlan oldalanként. Majd végleges könyvformában, addig így olvashatóbb.
\setlength{\oddsidemargin}{0.4cm}
\setlength{\evensidemargin}{0.4cm}


\ifthenelse
{\boolean{showComments}}
{
% Ideiglenes komment, a végleges változatban már nem lesz benne:
\newcommand{\wrk}[1]{\textcolor[gray]{.6}{(\textsl{#1})}}
}
{
% Ilyen, amikor tényleg nincs benne:
\newcommand{\wrk}[1]{}
}

\title{Bonyolultságelmélet feladatgyűjtemény}
\author{Varga Dániel} 
\date{2006.} 

\begin{document}

%\sloppy

\ifthenelse
{\boolean{compactForm}}
{

\newcommand{\pro}[1]{ \stepcounter{problem} \noindent \textbf{\theproblem .} #1 \vspace{2mm} }
\newcommand{\definition}[1]{\hspace*{-8mm} \textbf{Definíció:} #1 \vspace{2mm}}
\newcommand{\sol}[1]{\noindent \textbf{Megoldás:} #1 \vspace{2mm}}
\newcommand{\hint}[1]{\noindent \textbf{Segítség:} #1 \vspace{2mm}}
\newcommand{\prochat}[1]{#1}
\newcommand{\solchat}[1]{#1}

\input{complexity.txt}
}
% not compactForm
{

\maketitle

\tableofcontents

\part{Feladatok}

\newcommand{\pro}[1]{ \stepcounter{problem} \noindent \textbf{\theproblem .} #1 \vspace{4mm} }
\newcommand{\definition}[1]{\hspace*{-8mm} \textbf{Definíció:} #1 \vspace{2mm}}
\newcommand{\sol}[1]{\nothing}
\newcommand{\hint}[1]{\nothing}
\newcommand{\prochat}[1]{#1}
\newcommand{\solchat}[1]{\nothing}

\input{complexity.txt}

\part{Segítségek}

\setcounter{chapter}{0}

\renewcommand{\pro}[1]{ \stepcounter{problem} }
\renewcommand{\definition}[1]{\nothing}
\renewcommand{\sol}[1]{\nothing}
\renewcommand{\hint}[1]{\noindent \textbf{\theproblem .} #1 \vspace{2mm}}
\renewcommand{\prochat}[1]{\nothing}
\renewcommand{\solchat}[1]{\nothing}

\input{complexity.txt}

\part{Megoldások}

\setcounter{chapter}{0}

\renewcommand{\pro}[1]{ \begin{problem} #1 \end{problem} } % Figyelem, a Feladatok résszel ellentétben itt tétel-környezetet használok, ami következetlen, de a dőlt betű nem jön itt rosszul.
\renewcommand{\definition}[1]{\nothing}
\renewcommand{\sol}[1]{\noindent \textbf{Megoldás:} #1 \vspace{2mm}}
\renewcommand{\hint}[1]{\nothing}
\renewcommand{\prochat}[1]{\nothing}
\renewcommand{\solchat}[1]{#1}

\input{complexity.txt}

}
 
\end{document} 
